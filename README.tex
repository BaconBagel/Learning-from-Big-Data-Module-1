% Options for packages loaded elsewhere
\PassOptionsToPackage{unicode}{hyperref}
\PassOptionsToPackage{hyphens}{url}
%
\documentclass[
]{article}
\usepackage{amsmath,amssymb}
\usepackage{lmodern}
\usepackage{iftex}
\ifPDFTeX
  \usepackage[T1]{fontenc}
  \usepackage[utf8]{inputenc}
  \usepackage{textcomp} % provide euro and other symbols
\else % if luatex or xetex
  \usepackage{unicode-math}
  \defaultfontfeatures{Scale=MatchLowercase}
  \defaultfontfeatures[\rmfamily]{Ligatures=TeX,Scale=1}
\fi
% Use upquote if available, for straight quotes in verbatim environments
\IfFileExists{upquote.sty}{\usepackage{upquote}}{}
\IfFileExists{microtype.sty}{% use microtype if available
  \usepackage[]{microtype}
  \UseMicrotypeSet[protrusion]{basicmath} % disable protrusion for tt fonts
}{}
\makeatletter
\@ifundefined{KOMAClassName}{% if non-KOMA class
  \IfFileExists{parskip.sty}{%
    \usepackage{parskip}
  }{% else
    \setlength{\parindent}{0pt}
    \setlength{\parskip}{6pt plus 2pt minus 1pt}}
}{% if KOMA class
  \KOMAoptions{parskip=half}}
\makeatother
\usepackage{xcolor}
\usepackage[margin=1in]{geometry}
\usepackage{graphicx}
\makeatletter
\def\maxwidth{\ifdim\Gin@nat@width>\linewidth\linewidth\else\Gin@nat@width\fi}
\def\maxheight{\ifdim\Gin@nat@height>\textheight\textheight\else\Gin@nat@height\fi}
\makeatother
% Scale images if necessary, so that they will not overflow the page
% margins by default, and it is still possible to overwrite the defaults
% using explicit options in \includegraphics[width, height, ...]{}
\setkeys{Gin}{width=\maxwidth,height=\maxheight,keepaspectratio}
% Set default figure placement to htbp
\makeatletter
\def\fps@figure{htbp}
\makeatother
\setlength{\emergencystretch}{3em} % prevent overfull lines
\providecommand{\tightlist}{%
  \setlength{\itemsep}{0pt}\setlength{\parskip}{0pt}}
\setcounter{secnumdepth}{-\maxdimen} % remove section numbering
\ifLuaTeX
  \usepackage{selnolig}  % disable illegal ligatures
\fi
\IfFileExists{bookmark.sty}{\usepackage{bookmark}}{\usepackage{hyperref}}
\IfFileExists{xurl.sty}{\usepackage{xurl}}{} % add URL line breaks if available
\urlstyle{same} % disable monospaced font for URLs
\hypersetup{
  hidelinks,
  pdfcreator={LaTeX via pandoc}}

\author{}
\date{\vspace{-2.5em}}

\begin{document}

\hypertarget{github-repository-for-module-1-of-course-learning-from-big-data}{%
\section{GitHub Repository for Module 1 of Course ``Learning from Big
Data''}\label{github-repository-for-module-1-of-course-learning-from-big-data}}

\hypertarget{readme-content}{%
\subsection{README content}\label{readme-content}}

\begin{itemize}
\tightlist
\item
  \protect\hyperlink{repository-content}{Repository content}
\item
  \protect\hyperlink{Getting-up-to-speed}{Getting up to speed}
\item
  \protect\hyperlink{lecture-materials}{Lecture materials for Module 1}
\end{itemize}

\hypertarget{repository-content}{%
\subsection{Repository content}\label{repository-content}}

\begin{verbatim}
.
├── README.md            # this readme file
├── preparation          # a survival guide to get you familiar with Rmarkdown
└── lectures             # markdown files with the materials used in our sessions 
└── assignment 1         # the markdown file for the assignment of module 1 
└── data                 # all data files used in module 1, including data for the assignment, lectures, and tutorials
\end{verbatim}

The repository content is designed to make participation in Learning
from Big Data as easy and enjoyable for you as possible. The content of
this repo is being continuously updated until the first day of classes.

In the lectures, we will use R to illustrate implementation-related key
points. The files will be published in this repository ahead of the
lecture. Please make sure that you can execute the markdown file before
joining the class so you can easily follow the coding parts in the
lectures.

For the homework assignments, use the structure found in the final
assignment RMarkdown file, which will be soon available.

\hypertarget{getting-up-to-speed}{%
\subsection{Getting up to speed}\label{getting-up-to-speed}}

If you are not familiar with R markdown please go through this short
tutorial in the following link.

\begin{itemize}
\tightlist
\item
  \href{https://github.com/guiliberali/Learning-from-Big-Data-Module-1/blob/main/preparation/Intro-to-RMarkdown.pdf}{Introduction
  to R and R markdown}
\end{itemize}

Greta Wackerbauer, our TA
(\href{mailto:learningbigdata22@gmail.com}{\nolinkurl{learningbigdata22@gmail.com}})
will review and discuss this short tutorial on Rmarkdown on September 2
at 3pm (i.e., right after the first lecture) and see if anyone has any
questions on how to use R and RMarkdown. Feel free to attend if you
would like to get some tips, but do make sure you first try to install R
studio and R Markdown on your computer before the class.

\hypertarget{lecture-materials}{%
\subsection{Lecture materials}\label{lecture-materials}}

Here are links to theory and practice (data/code) when applicable. You
may need to use your school VPN to get access to published papers.

Theory - Lecture 01-1: Machine learning foundations and natural language
processing:
\href{https://journals.sagepub.com/doi/full/10.1177/0022242919873106}{here}
and
\href{https://www.thecasecentre.org/course/registerForCourse?ucc=C\%2D4874\%2D6030\%2DSCH}{the
Tamarin case can be downloaded here}

Code/data:\\
-
\href{https://github.com/guiliberali/Learning-from-Big-Data-Module-1/blob/main/lectures/Lecture_1/session_1.Rmd}{Lecture
01-1}.

-\href{https://github.com/guiliberali/Learning-from-Big-Data-Module-1/blob/main/lectures/Lecture_2/Cartoon_Example_NBC.xlsx}{Lecture
01-2}. Code/data will be here.

\begin{itemize}
\item
  Lecture 01-3: code/data will be here.
\item
  Tutorial Assignment 1
\end{itemize}

\end{document}
